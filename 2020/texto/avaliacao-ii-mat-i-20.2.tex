\documentclass[12pt,a4paper]{article}
\usepackage[utf8]{inputenc}
\usepackage[brazil]{babel}
\usepackage{graphicx}
\usepackage{amssymb, amsfonts, amsmath}
\usepackage{float}
\usepackage{enumerate}
\usepackage[top=1.5cm, bottom=1.5cm, left=1.25cm, right=1.25cm]{geometry}

\begin{document}
\pagestyle{empty}

\begin{center}
  \begin{tabular}{ccc}
    \begin{tabular}{c}
      \includegraphics[scale=0.25]{../../biblioteca/imagem/brasao-de-armas-brasil} \\
    \end{tabular} & 
    \begin{tabular}{c}
      Ministério da Educação \\
      Universidade Federal dos Vales do Jequitinhonha e Mucuri \\
      Faculdade de Ciências Sociais, Aplicadas e Exatas - FACSAE \\
      Departamento de Ciências Exatas - DCEX \\
      Disciplina: Matemática Elementar I \quad Semestre: 2020/2\\
      Prof. Me. Luiz C. M. de Aquino\\
    \end{tabular} &
    \begin{tabular}{c}
      \includegraphics[scale=0.25]{../../biblioteca/imagem/logo-ufvjm} \\
    \end{tabular}
  \end{tabular}
\end{center}

\begin{center}
 \textbf{Avaliação II}
\end{center}

\textbf{Instruções}
\begin{itemize}
 \item Todas as justificativas necessárias na solução de cada questão devem estar presentes nesta avaliação;
 \item As respostas finais de cada questão devem estar escritas de caneta;
 \item Esta avaliação tem um total de 35,0 pontos.
\end{itemize}

\begin{enumerate}
    \item \textbf{[7,0 pontos]} Considerando $f(x) = \dfrac{1}{7}x - 5$ e $g(x) = 4x - \dfrac{3}{2}$, determine as composições
    abaixo.
    \begin{enumerate}
      \item $f\circ g$
      \item $g\circ f$
    \end{enumerate}

    \item \textbf{[7,0 pontos]} Se $f\circ g(x) = \dfrac{5}{8}x - 1$ e $f(x) = \dfrac{1}{5}x + 4$,
    determine a expressão de $g$.
  
    \item \textbf{[7,0 pontos]} Se $f\circ g(x) = \dfrac{4 - 7x}{3}$ e $g(x) = \dfrac{1}{5}x + 1$,
    determine a expressão de $f$.

    \item \textbf{[7,0 pontos]} Determine o valor da constante $c$ de tal forma que
    $f\circ g(x) = g\circ f(x)$, onde $f(x) = 2x + c$ e 
    $g(x) = cx + 6$.

    \item \textbf{[7,0 pontos]} Supondo que $f(x) = \dfrac{35 - 3x}{5}$, determine a função
  $g$ tal que $f\circ g(x) = x$.

\end{enumerate}

\end{document}
\documentclass[12pt,a4paper]{article}
\usepackage[utf8]{inputenc}
\usepackage[brazil]{babel}
\usepackage{graphicx}
\usepackage{amssymb, amsfonts, amsmath}
\usepackage{float}
\usepackage{enumerate}
\usepackage[top=1.5cm, bottom=1.5cm, left=1.25cm, right=1.25cm]{geometry}

\begin{document}
\pagestyle{empty}

\begin{center}
  \begin{tabular}{ccc}
    \begin{tabular}{c}
      \includegraphics[scale=0.25]{../../biblioteca/imagem/brasao-de-armas-brasil} \\
    \end{tabular} & 
    \begin{tabular}{c}
      Ministério da Educação \\
      Universidade Federal dos Vales do Jequitinhonha e Mucuri \\
      Faculdade de Ciências Sociais, Aplicadas e Exatas - FACSAE \\
      Departamento de Ciências Exatas - DCEX \\
      Disciplina: Matemática Elementar I \quad Semestre: 2020/2\\
      Prof. Me. Luiz C. M. de Aquino\\
    \end{tabular} &
    \begin{tabular}{c}
      \includegraphics[scale=0.25]{../../biblioteca/imagem/logo-ufvjm} \\
    \end{tabular}
  \end{tabular}
\end{center}

\begin{center}
 \textbf{Avaliação I}
\end{center}

\textbf{Instruções}
\begin{itemize}
 \item Todas as justificativas necessárias na solução de cada questão devem estar presentes nesta avaliação;
 \item As respostas finais de cada questão devem estar escritas de caneta;
 \item Esta avaliação tem um total de 30,0 pontos.
\end{itemize}

\begin{enumerate}
  \item \textbf{[6,0 pontos]} Considere os conjuntos $A = \left\{-4,\,-3,\,8,\,9\right\}$,
    $B = \left\{x\in\mathbb{Z} \,|\, x < 7\right\}$ e
    $C = \left\{x\in\mathbb{N} \,|\, x > 5\right\}$.

    Determine:
    \begin{enumerate}
      \item $A\cup B$
      \item $B\cap C$
      \item $B\setminus A$
      \item $(B\cup C)\cap A$
    \end{enumerate}
  
  \item \textbf{[6,0 pontos]} Suponha que $A = \{x\in\mathbb{N} \,|\, x^2 - x - 6 = 0\}$ e 
    $B = \{x\in\mathbb{Z} \,|\, -5 \leq x < 6 \}$. Classifique em Verdadeiro ou
    Falso:

    \begin{enumerate}[(\ \ )]
      \item $A\cap B = \{-2,\,3\}$
      \item $n(B) = 10$
      \item $A\cup B = A$
      \item $A \neq \varnothing$
      \item $A - B = \varnothing$
    \end{enumerate}
  
  \item \textbf{[6,0 pontos]} Determine os elementos de cada relação binária abaixo,
    considerando que $A = \{-2,\,0,\,2\}$ e $B = \{-5,\, 0,\, 4\}$.
    \begin{enumerate}
      \item $R = \{(x,\,y) \in A\times B \,|\, y = x - 3\}$
      \item $R = \left\{(x,\,y) \in A\times B \,|\, y = \dfrac{9}{2}x - 5\right\}$
      \item $R = \left\{(x,\,y) \in B\times A \,|\, y = \dfrac{x - 4}{2}\right\}$
      \item $R = \left\{(x,\,y) \in B\times B \,|\, y = - 1 - x\right\}$
    \end{enumerate}

  \item \textbf{[6,0 pontos]} Sejam os conjuntos $A = \{x\in\mathbb{Z}\,|\, -2 < x < 3\}$ e 
    $B = \{x\in\mathbb{Z}\,|\, -3 < x \leq 2\}$. Represente graficamente os
    produtos cartesianos abaixo.
    \begin{enumerate}
      \item $A\times B$
      \item $B\times A$
      \item $A\times A$
      \item $B\times B$
    \end{enumerate}

  \item \textbf{[6,0 pontos]} Seja $f:A\to B$ dada por $f(x) = 2x - 1$, sendo que
    $A = \{-1,\,0,\,1\}$ e $B = \{-4,\,-3,-1,\,0,\,\,1,\,5\}$.
    Identifique o domínio, o contradomínio e a imagem de $f$.

\end{enumerate}

\end{document}
\documentclass[12pt,a4paper]{article}
\usepackage[utf8]{inputenc}
\usepackage[brazil]{babel}
\usepackage{graphicx}
\usepackage{amssymb, amsfonts, amsmath}
\usepackage{float}
\usepackage{enumerate}
\usepackage[top=1.5cm, bottom=1.5cm, left=1.25cm, right=1.25cm]{geometry}

\begin{document}
\pagestyle{empty}

\begin{center}
  \begin{tabular}{ccc}
    \begin{tabular}{c}
      \includegraphics[scale=0.25]{../../biblioteca/imagem/brasao-de-armas-brasil} \\
    \end{tabular} & 
    \begin{tabular}{c}
      Ministério da Educação \\
      Universidade Federal dos Vales do Jequitinhonha e Mucuri \\
      Faculdade de Ciências Sociais, Aplicadas e Exatas - FACSAE \\
      Departamento de Ciências Exatas - DCEX \\
      Disciplina: Matemática I \quad Semestre: 2020/1\\
      Prof. Me. Luiz C. M. de Aquino\\
    \end{tabular} &
    \begin{tabular}{c}
      \includegraphics[scale=0.25]{../../biblioteca/imagem/logo-ufvjm} \\
    \end{tabular}
  \end{tabular}
\end{center}

\begin{center}
  \textbf{Lista IV}
\end{center}

\begin{enumerate}
  \item Determine a raiz das funções dadas por:
    \begin{enumerate}
      \item $f(x) = 2^{x - 2} - 4\cdot 2^x$
      \item $g(x) = \dfrac{1}{2^{1-x}} - 16^x$
      \item $h(x) = \log^2_3 x + \dfrac{3}{4}\log_3 x - \dfrac{1}{4}$
      \item $j(x) = \log_2 (x^2 - 1) - 3$
    \end{enumerate}     

  \item Uma pessoa tomou 60 mg de certo remédio. A bula do remédio informava que
    a cada 6 horas a sua quantidade no organismo reduzia-se a metade. Com base
    nessas informações, para que a quantidade no organismo atinja 7,5 mg, qual
    é o tempo (em horas) necessário?

  \item A escala Richter -- assim chamada em homenagem ao sismólogo americano
  Charles F. Richter -- mede a magnitude de um terremoto em uma escala logarítmica
  de base $10$. A intensidade $I$ de um terremoto medida nessa escala, dada por um
  valor entre $0$ e $8,9$ (para o maior terremoto conhecido), é obtida pela
  fórmula $I=\dfrac{2}{3}\log_{10}\dfrac{E}{E_{0}}$, em que $E$ é a energia
  (em kWh -- quilowatts-hora) liberada pelo terremoto e $E_{0}=7\cdot10^{-3}$ kWh.
  Com base nessas informações, responda os quesitos abaixo.
    \begin{enumerate}
      \item Um terremoto de intensidade $8$ libera quanta energia?
      \item Se a energia calculada no quesito (a) pudesse ser convertida para
        energia elétrica e usada para abastecer uma cidade que consome $3,5\cdot10^{5}$
        kWh por dia, então quantos dias a cidade ficaria abastecida?
    \end{enumerate}
 
  \item Sobre o gráfico das funções definidas por $f(x)=2^x$ e $g(x) = x^2$, podemos afirmar que:
    \begin{enumerate}
      \item possuem dois pontos de interseção quando $x\in\left[-\dfrac{1}{2};\,3\right]$.
      \item possuem algum ponto de interseção quando $x\in[0;\,1]$.
      \item não possuem ponto de interseção quando $x\in[-1;\,1]$.
      \item possuem algum ponto de interseção quando $x\in[-1;\,0]$. %<==
      \item possuem dois pontos de interseção quando $x\in\left[-3;\,\dfrac{1}{2}\right]$.
    \end{enumerate}

  \item Considere as funções definidas por $f(x) = \log_a x$ e $g(x) = \log_b x$, onde $a\neq b$.
  Se $n\in\mathbb{N}^*$, podemos afirmar que:
    \begin{enumerate}
      \item $f\left(a^n\right) + g\left(\dfrac{1}{b^n}\right) = 0$. %<==
      \item $f\left(b^n\right) - g\left(a^n\right) = 0$.
      \item $\left(\dfrac{f(a)}{g(b)}\right)^n = f(a)^n - g(b)^n$.
      \item $f\left(a^n\right)g\left(b^n\right) = [f(a) + g(b)]^n$.
      \item $f\left(\sqrt[n]{a}\right)\sqrt[n]{g(b)} = n$.
    \end{enumerate}

  \item A soma das raízes da função definida por $f(x) = 4^x - 2^{x+3} + 15$ é igual a:
    \begin{enumerate}
      \item $\log_2 8$.
      \item $\log_2 4$.
      \item $\log_2 3$.
      \item $\log_2 15$. %<==
      \item $\log_2 5$.
    \end{enumerate}
    
  \item Seja a função $f:\mathbb{N}\to\mathbb{R}$ definida por:
  $$f(x)=\begin{cases}10;\, x=0 \\ 10\log (f(x-1));\,x>0 \end{cases}.$$
%log (10 log(10 log(10 log(10))))

  É correto afirmar que $f(100)$ é igual a:

  \begin{enumerate}
    \item $10$. %<==
    \item $10^2$.
    \item $10^3$.
    \item $10^4$.
    \item $10^5$.
  \end{enumerate}
  
  \item Classifique cada afirmação abaixo como Verdadeiro ou Falso.
    \begin{enumerate}[(\ \ )]
      \item Se $a\in \mathbb{R}$, então $a^0 = 1$.
      \item $0^n = 0$, para qualquer $n\in \mathbb{R}$.
      \item Se $a^x = 2$ e $a^y = 3$, então $a^{x + y} = 6$.
      \item $\sqrt[3]{\sqrt[4]{x}} = \sqrt[12]{x}$
      \item Se $\log_b a = 2$ e $\log_b c = 3$, então $\log_b ac = 5$.
      \item Se $\log_b a = 10$ e $\log_b c = 2$, então $\log_b \dfrac{a}{c} = 5$.
      \item Se $\log_b a = 4$, então $a = b^4$.
      \item $\log_b a + \log_b a = \log_b a^2$
      \item $\log_2 8 - \log_8 2 = 0$
      \item $\log_{30} 27000 - \log_{20} 8000 = 0$
    \end{enumerate}

\end{enumerate}

\end{document}
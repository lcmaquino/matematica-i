\documentclass[12pt,a4paper]{article}
\usepackage[utf8]{inputenc}
\usepackage[brazil]{babel}
\usepackage{graphicx}
\usepackage{amssymb, amsfonts, amsmath}
\usepackage{float}
\usepackage{enumerate}
\usepackage[top=1.5cm, bottom=1.5cm, left=1.25cm, right=1.25cm]{geometry}

\begin{document}
\pagestyle{empty}

\begin{center}
  \begin{tabular}{ccc}
    \begin{tabular}{c}
      \includegraphics[scale=0.25]{../../biblioteca/imagem/brasao-de-armas-brasil} \\
    \end{tabular} & 
    \begin{tabular}{c}
      Ministério da Educação \\
      Universidade Federal dos Vales do Jequitinhonha e Mucuri \\
      Faculdade de Ciências Sociais, Aplicadas e Exatas - FACSAE \\
      Departamento de Ciências Exatas - DCEX \\
      Disciplina: Matemática Elementar I \quad Semestre: 2020/2\\
      Prof. Me. Luiz C. M. de Aquino\\
    \end{tabular} &
    \begin{tabular}{c}
      \includegraphics[scale=0.25]{../../biblioteca/imagem/logo-ufvjm} \\
    \end{tabular}
  \end{tabular}
\end{center}

\begin{center}
  \textbf{Lista IV}
\end{center}

\begin{enumerate}
  \item Considerando $f(x) = 2x - 5$ e $g(x) = 3x + 8$, determine as composições
    abaixo.
    \begin{enumerate}
      \item $f\circ g$
      \item $g\circ f$
      \item $f\circ f$
      \item $g\circ g$
    \end{enumerate}

  \item Se $f\circ g(x) = \dfrac{5}{7}x - 2$ e $f(x) = 4x + \dfrac{1}{5}$,
    determine a expressão de $g$.
  
  \item Se $f\circ g(x) = \dfrac{2 - 3x}{5}$ e $g(x) = \dfrac{1}{2}x + 8$,
    determine a expressão de $f$.

  \item Determine o valor da constante $c$ de tal forma que
    $f\circ g(x) = g\circ f(x)$, onde $f(x) = 2x + c$ e 
    $g(x) = cx + 6$

  \item Supondo que $f(x) = 5x - \dfrac{1}{2}$, determine a função 
  $g$ tal que $f\circ g(x) = x$.

  \item Prove que se $f(x) = ax + b$ (com $a\neq 0$) e 
    $f\circ g(x) = x$, então $g\circ f(x) = x$.
      
\end{enumerate}

\begin{center}
  \textbf{Gabarito}
\end{center}

[1] (a) $f\circ g(x) = 6x + 11$. (b) $g\circ f(x) = 6x - 7$. 
(c) $f\circ f(x) = 4x - 15$. (d) $g\circ g(x) = 9x + 32$. 
[2] $g(x) = \dfrac{5}{28}x - \dfrac{11}{20}$. 
[3] $f(x) = -\dfrac{6}{5}x + 10$. 
[4] $c = -2$ ou $c = 3$. 
[5] $g(x) = \dfrac{2x + 1}{10}$.
[6] Sugestão: use $f(x) = ax + b$ e $f\circ g(x) = x$ para determinar $g(x)$.
Em seguida, calcule $g\circ f(x)$. 

\end{document}
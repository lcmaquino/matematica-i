\documentclass[12pt,a4paper]{article}
\usepackage[utf8]{inputenc}
\usepackage[brazil]{babel}
\usepackage{graphicx}
\usepackage{amssymb, amsfonts, amsmath}
\usepackage{float}
\usepackage{enumerate}
\usepackage[top=1.5cm, bottom=1.5cm, left=1.25cm, right=1.25cm]{geometry}

\begin{document}
\pagestyle{empty}

\begin{center}
  \begin{tabular}{ccc}
    \begin{tabular}{c}
      \includegraphics[scale=0.25]{../../biblioteca/imagem/brasao-de-armas-brasil} \\
    \end{tabular} & 
    \begin{tabular}{c}
      Ministério da Educação \\
      Universidade Federal dos Vales do Jequitinhonha e Mucuri \\
      Faculdade de Ciências Sociais, Aplicadas e Exatas - FACSAE \\
      Departamento de Ciências Exatas - DCEX \\
      Disciplina: Matemática I \quad Semestre: 2020/1\\
      Prof. Me. Luiz C. M. de Aquino\\
    \end{tabular} &
    \begin{tabular}{c}
      \includegraphics[scale=0.25]{../../biblioteca/imagem/logo-ufvjm} \\
    \end{tabular}
  \end{tabular}
\end{center}

\begin{center}
 \textbf{Avaliação III}
\end{center}

\textbf{Instruções}
\begin{itemize}
 \item Todas as justificativas necessárias na solução de cada questão devem estar presentes nesta avaliação;
 \item As respostas finais de cada questão devem estar escritas de caneta;
 \item Esta avaliação tem um total de 35,0 pontos.
\end{itemize}

\begin{enumerate}
  \item \textbf{[7,0 pontos]} Determine a raiz das funções dadas por:
    \begin{enumerate}
      \item $g(x) = 9^{x-\frac{1}{2}} - 242\cdot 3^{x-2} - 9$
      \item $j(x) = 16\log_2^2 x - 6\log_2 x - 1$
    \end{enumerate}
  
  \item \textbf{[7,0 pontos]} A escala Richter -- assim chamada em homenagem ao sismólogo americano
  Charles F. Richter -- mede a magnitude de um terremoto em uma escala logarítmica
  de base $10$. A intensidade $I$ de um terremoto medida nessa escala, dada por um
  valor entre $0$ e $8,9$ (para o maior terremoto conhecido), é obtida pela
  fórmula $I=\dfrac{2}{3}\log_{10}\dfrac{E}{E_{0}}$, em que $E$ é a energia
  (em kWh -- quilowatts-hora) liberada pelo terremoto e $E_{0}=7\cdot10^{-3}$ kWh.
  Com base nessas informações, responda os quesitos abaixo.
    \begin{enumerate}
      \item Se um terremoto liberou $7\cdot 10^6$ kWh de energia, então qual foi a intensidade dele?
      \item Se um terremoto tem intensidade $4$, então quanta energia ele liberou?
    \end{enumerate}
  
  \item \textbf{[7,0 pontos]} Um capital $C$ em regime de juros compostos com 
  taxa percentual $i$ (ao mês), gera um montante $M$ após o tempo $t$ (em meses) que é dado por:
  $$M = C(1 + i)^t$$

  Determine o tempo necessário para um capital de R\$ 110,00 gerar um montante
  de R\$ 146,41 em regime de juros compostos com taxa de 21\% ao mês.
  
  \item \textbf{[7,0 pontos]} Suponha que $x$ e $y$ são números reais tais que:
  $$\begin{cases} 
    \log_{8} (x+y) = 1\\
    \log_{13} x + \log_{13} y = 1
  \end{cases}$$
  
  Determine os números $x$ e $y$.

  \item \textbf{[7,0 pontos]} Determine o ponto de interseção entre os gráficos
    das funções dadas por \mbox{$f(x) = 6\log_9^2 x$} e $g(x) = 3 + 3\log_9 x$.
  
\end{enumerate}

\end{document}
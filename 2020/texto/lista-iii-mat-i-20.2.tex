\documentclass[12pt,a4paper]{article}
\usepackage[utf8]{inputenc}
\usepackage[brazil]{babel}
\usepackage{graphicx}
\usepackage{amssymb, amsfonts, amsmath}
\usepackage{float}
\usepackage{enumerate}
\usepackage[top=1.5cm, bottom=1.5cm, left=1.25cm, right=1.25cm]{geometry}

\begin{document}
\pagestyle{empty}

\begin{center}
  \begin{tabular}{ccc}
    \begin{tabular}{c}
      \includegraphics[scale=0.25]{../../biblioteca/imagem/brasao-de-armas-brasil} \\
    \end{tabular} & 
    \begin{tabular}{c}
      Ministério da Educação \\
      Universidade Federal dos Vales do Jequitinhonha e Mucuri \\
      Faculdade de Ciências Sociais, Aplicadas e Exatas - FACSAE \\
      Departamento de Ciências Exatas - DCEX \\
      Disciplina: Matemática Elementar I \quad Semestre: 2020/2\\
      Prof. Me. Luiz C. M. de Aquino\\
    \end{tabular} &
    \begin{tabular}{c}
      \includegraphics[scale=0.25]{../../biblioteca/imagem/logo-ufvjm} \\
    \end{tabular}
  \end{tabular}
\end{center}

\begin{center}
  \textbf{Lista III}
\end{center}

\begin{enumerate}
  \item Seja $f:A\to B$ dada por $f(x) = 2x - 1$, sendo que
    $A = \{-1,\,0,\,1\}$ e $B = \{-4,\,-3,-1,\,0,\,\,1,\,5\}$.
    Identifique o domínio, o contradomínio e a imagem de $f$.

  \item Considere a função $f:\mathbb{Z}\to\mathbb{R}$ definida por $f(x)=\dfrac{x - 1}{x^2 + 1}$. 
    Complete a tabela abaixo.

    \begin{table}[H]
      \centering
      \begin{tabular}{c|c}
          $x$ & $f(x)$ \\ \hline
          $-2$ &  \\ \hline
          $-1$ &  \\ \hline
          $0$ &  \\ \hline
          $1$ &  \\ \hline
          $2$ &
      \end{tabular}
    \end{table}
    
  \item Seja uma função real $f$ tal que $f(a + b) = f(a) + f(b)$, para 
  todo $a$ e $b$ real. Se $f(4) = 5$, então determine $f(1)$.

  \item Classifique cada afirmação abaixo como Verdadeiro ou Falso.
    \begin{enumerate}[(a)]
      \item É possível existir uma função $f$ tal que $f(4) = 2$ e $f(4) = -2$.
      \item É possível existir uma função $f$ tal que $f(2) = 4$ e $f(-2) = 4$.
      \item Se $f(2) = 4$ e $f(3) = 6$, então $f(2 + 3) = 4 + 6$.
      \item Se $f(2) = 5$ e $f(5) = 10$, então $f(f(2)) = 10$.
      \item Se $f(2) = 5$ e $f(5) = 10$, então $f(f(5)) = 10$.
      \item Se $a$ é um elemento do domínio de $f$, então existe um elemento $b$
        do contradomínio de $f$ tal que $f(a) = b$.
      \item Se $b$ é um elemento do contradomínio de $f$, então existe um elemento $a$
        do domínio de $f$ tal que $f(a) = b$.
      \item Se $b$ é um elemento da imagem de $f$, então existe um elemento $a$
        do domínio de $f$ tal que $f(a) = b$.
      \item Todo elemento do contradomínio é também um elemento da imagem.
      \item Todo elemento da imagem é também um elemento do contradomínio.
    \end{enumerate}
    
\end{enumerate}

\begin{center}
  \textbf{Gabarito}
\end{center}

[1] Domínio: $A$. Contradomínio: $B$. Imagem: $\{-3,\,-1,\,1\}$. 
[2] $f(-2) = -\dfrac{3}{5}$, $f(-1) = -1$, $f(0) = -1$, $f(1) = 0$, $f(2) = \dfrac{1}{5}$. 
[3] $f(1) = \dfrac{5}{4}$. 
[4] (a) F. (b) V. (c) F. (d) V. (e) F. (f) V. (g) F. (h) V. (i) F. (j) V. 

\end{document}
\documentclass[12pt,a4paper]{article}
\usepackage[utf8]{inputenc}
\usepackage[brazil]{babel}
\usepackage{graphicx}
\usepackage{amssymb, amsfonts, amsmath}
\usepackage{float}
\usepackage{enumerate}
\usepackage[top=2.5cm, bottom=2.5cm, left=1.25cm, right=1.25cm]{geometry}

\begin{document}
\pagestyle{empty}

\begin{center}
  \begin{tabular}{ccc}
    \begin{tabular}{c}
      \includegraphics[scale=0.25]{../../biblioteca/imagem/brasao-de-armas-brasil} \\
    \end{tabular} & 
    \begin{tabular}{c}
      Ministério da Educação \\
      Universidade Federal dos Vales do Jequitinhonha e Mucuri \\
      Faculdade de Ciências Sociais, Aplicadas e Exatas - FACSAE \\
      Departamento de Ciências Exatas - DCEX \\
      Disciplina: Matemática Elementar I \quad Semestre: 2021/1\\
      Prof. Me. Luiz C. M. de Aquino\\
    \end{tabular} &
    \begin{tabular}{c}
      \includegraphics[scale=0.25]{../../biblioteca/imagem/logo-ufvjm} \\
    \end{tabular}
  \end{tabular}
\end{center}

\begin{center}
  \textbf{Lista VI}
\end{center}

\begin{enumerate}
  \item Sejam as funções dadas por $f(x) = 2x - 3$ e $g(x) = x^2 - 4x + 1$.
  Determine as composições:
    \begin{enumerate}
      \item $f\circ g$
      \item $g\circ f$
      \item $f\circ f$
      \item $g\circ g$
    \end{enumerate}

  \item Se $f\circ g(x) = \dfrac{5}{8}x - 6$ e $f(x) = -3x + \dfrac{1}{7}$,
    determine a expressão de $g$.
  
  \item Se $f\circ g(x) = \dfrac{-3 + 2x}{6}$ e $g(x) = \dfrac{1}{2}x - 4$,
    determine a expressão de $f$.

  \item Determine a função inversa de cada função $f$ dada abaixo.
    \begin{enumerate}
      \item $f(x) = 2x - \dfrac{1}{4}$
      \item $f(x) = \dfrac{2}{5}x + 6$
      \item $f(x) = x^2 - 2x + 3$, $x \in (-\infty,\,1)$.
      \item $f(x) = -x^2 + 4x + 5$, $x \in (2,\,+\infty)$.
    \end{enumerate}

\end{enumerate}

\begin{center}
  \textbf{Gabarito}
\end{center}

[1] (a) $f\circ g(x) = 2x^2 - 8x - 1$. (b) $g\circ f(x) = 4x^2 - 20x + 22$. 
(c) $f\circ f(x) = 4x - 9$. (d) $g\circ g(x) = x^4 - 8x^3 + 14x^2 + 8x - 2$. 
[2] $g(x) = -\dfrac{5}{24}x + \dfrac{43}{21}$. 
[3] $f(x) = \dfrac{4x  + 13}{6}$. 
[4] (a) $f^{-1}(x) = \dfrac{x}{2} - \dfrac{1}{8}$. 
(b) $f^{-1}(x) = \dfrac{5x}{2} - 15$.
(c) $f^{-1}(x) = 1 - \sqrt{x - 2}$. 
(d) $f^{-1}(x) = 2 + \sqrt{9 - x}$. 
\end{document}
\documentclass[12pt,a4paper]{article}
\usepackage[utf8]{inputenc}
\usepackage[brazil]{babel}
\usepackage{graphicx}
\usepackage{amssymb, amsfonts, amsmath}
\usepackage{float}
\usepackage{enumerate}
\usepackage[top=1.5cm, bottom=1.5cm, left=1.25cm, right=1.25cm]{geometry}

\begin{document}
\pagestyle{empty}

\begin{center}
  \begin{tabular}{ccc}
    \begin{tabular}{c}
      \includegraphics[scale=0.25]{../../biblioteca/imagem/brasao-de-armas-brasil} \\
    \end{tabular} & 
    \begin{tabular}{c}
      Ministério da Educação \\
      Universidade Federal dos Vales do Jequitinhonha e Mucuri \\
      Faculdade de Ciências Sociais, Aplicadas e Exatas - FACSAE \\
      Departamento de Ciências Exatas - DCEX \\
      Disciplina: Matemática Elementar I \quad Semestre: 2021/1\\
      Prof. Me. Luiz C. M. de Aquino\\
    \end{tabular} &
    \begin{tabular}{c}
      \includegraphics[scale=0.25]{../../biblioteca/imagem/logo-ufvjm} \\
    \end{tabular}
  \end{tabular}
\end{center}

\begin{center}
 \textbf{Avaliação III}
\end{center}

\textbf{Instruções}
\begin{itemize}
 \item Todas as justificativas necessárias na solução de cada questão devem estar presentes nesta avaliação;
 \item As respostas finais de cada questão devem estar escritas de caneta;
 \item Esta avaliação tem um total de 35,0 pontos.
\end{itemize}

\begin{enumerate}
  \item \textbf{[7,0 pontos]} Sejam as funções dadas por $f(x) = -6x + 4$ e $g(x) = x^2 - 5x + 2$.
  Determine as composições:
    \begin{enumerate}
      \item $f\circ g$
      \item $g\circ f$
    \end{enumerate}

  \item \textbf{[7,0 pontos]} Se $f\circ g(x) = \dfrac{-3 + 2x}{6}$ e $g(x) = \dfrac{1}{4}x - 2$,
    determine a expressão de $f$.

  \item \textbf{[7,0 pontos]} Determine a função inversa de cada função $f$ dada abaixo.
    \begin{enumerate}
      \item $f(x) = -\dfrac{6}{7}x + 5$
      \item $f(x) = -x^2 + 6x + 5$, $x \in (3,\,+\infty)$.
    \end{enumerate} 

  \item \textbf{[7,0 pontos]} Dadas as funções definidas por $f(x) = -8x + 2$ e $g(x) = ax + b$, 
  se $f\circ g(x) = 8x + 12$, então calcule o valor de $a + b$.

  \item \textbf{[7,0 pontos]} Considerando que $f(x) = -3x + 4$ e $g^{-1}(x) = \dfrac{x - m}{5}$,
   determine o valor de $m$ sabendo que $f\circ g(x) = g\circ f(x)$.

\end{enumerate}

\end{document}
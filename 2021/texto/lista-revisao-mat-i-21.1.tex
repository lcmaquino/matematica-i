\documentclass[12pt,a4paper]{article}
\usepackage[utf8]{inputenc}
\usepackage[brazil]{babel}
\usepackage{graphicx}
\usepackage{amssymb, amsfonts, amsmath}
\usepackage{float}
\usepackage{enumerate}
\usepackage[top=2.5cm, bottom=2.5cm, left=1.25cm, right=1.25cm]{geometry}

\begin{document}
\pagestyle{empty}

\begin{center}
  \begin{tabular}{ccc}
    \begin{tabular}{c}
      \includegraphics[scale=0.25]{../../biblioteca/imagem/brasao-de-armas-brasil} \\
    \end{tabular} & 
    \begin{tabular}{c}
      Ministério da Educação \\
      Universidade Federal dos Vales do Jequitinhonha e Mucuri \\
      Faculdade de Ciências Sociais, Aplicadas e Exatas - FACSAE \\
      Departamento de Ciências Exatas - DCEX \\
      Disciplina: Matemática Elementar I \quad Semestre: 2021/1\\
      Prof. Me. Luiz C. M. de Aquino\\
    \end{tabular} &
    \begin{tabular}{c}
      \includegraphics[scale=0.25]{../../biblioteca/imagem/logo-ufvjm} \\
    \end{tabular}
  \end{tabular}
\end{center}

\begin{center}
  \textbf{Lista de Revisão}
\end{center}

\begin{enumerate}
  \item Considere os conjuntos $A = \left\{-2,\,-1,\,0,\,1\right\}$,
    $B = \left\{x\in\mathbb{Z} \,|\, x < 1\right\}$ e
    $C = \left\{x\in\mathbb{N} \,|\, x > 1\right\}$.

    Determine:
    \begin{enumerate}
      \item $A\cap C$
      \item $B\cap C$
      \item $A\setminus C$
      \item $A\setminus B$
    \end{enumerate}
    

  \item Seja $f:A\to B$ dada por $f(x) = \dfrac{3}{2}x + \dfrac{1}{2}$, sendo que:
    $$A = \{-1,\,0,\,1\},\quad B = \left\{-2,-1,-\frac{1}{2},\,0,\,\frac{1}{2},\,\,1,\,2\right\}$$
    Identifique o domínio, o contradomínio e a imagem de $f$.
  
  \item Seja uma função real $f$ tal que $f(x) = 1 - \dfrac{4x}{(x+1)^2}$.
  Determine o produto $f(x)f(-x)$.
  
  \item Classifique cada afirmação abaixo como Verdadeiro ou Falso.
    \begin{enumerate}[(a)]
      \item É possível existir uma função $f$ tal que $f(8) = 2$ e $f(2) = 8$.
      \item É possível existir uma função $f$ tal que $f(-2) = 5$ e $f(-2) = 5$.
      \item Se $f(2) = 4$ e $f(3) = 6$, então $f(2 - 3) = 4 - 6$.
      \item Todo elemento do domínio é também um elemento da imagem.
      \item Todo elemento da imagem é também um elemento do contradomínio.
    \end{enumerate}
    
\end{enumerate}

\end{document}